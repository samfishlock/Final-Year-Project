\subsection{Background}
Since Shor first introduced a polynomial time algorithm for prime factorisation and discrete logarithms~\cite{shor1999polynomial}, there has been a lot of interest in quantum computing and being able to create a working set of hardware which realises Shor's theoretical algorithm. Perhaps the largest difficulty faced when trying to create a quantum computer is quantum decoherence~\cite{ponnath2006difficulties} - this is when a qubit interacts with its surroundings and leads to a collapse of the superposition. Quantum decoherence becomes increasingly hard to eliminate when more qubits are added to the processor. Despite this obstacle, there have been several quantum computers manufactured: in May 2016, IBM released the IBM Q Experience; a five qubit quantum processor, and in May 2017, they expanded this to add a 16-qubit processor which has been shown to be able to become fully entangled~\cite{wang201816}. At present, the largest functional quantum computer, named Bristlecone, has 72 Qubits and is made by Google~\cite{bristlecone2018}.
\subsection{Appliances of Quantum Computing}
Quantum computers have been shown to be faster than classical computers at solving certain problems~\cite{Bravyi308}, and one of these areas which is of particular importance is the area of cryptography. Common cryptographic schemes used widely today implement a function which, when applied to the text, is unable to be reversed. One such example of a scheme which follows this is RSA, which relies on the fact that it is very hard for classical computers to perform prime factorisation when the prime factors for that number are large. The most efficient classical algorithm to break RSA is the General Number Field Sieve~\cite{GNFS2006}, which is sub-exponential in complexity, Using Shor's algorithm, we can break RSA in bounded-error quantum polynomial time (BQP), which is almost exponentially faster than the General Number Field Sieve.
Although the theory suggests that quantum computing can be very powerful, the hardware is not currently there to support it - for example the largest number factored by Shor's Algorithm is 21, which required 10 qubits~\cite{21factorshors2012}, however the largest number factorised on a quantum computer is 56,153, which required 4 qubits~\cite{dattani2014quantum}. These are still far off from classical computing capability, for scale, the largest number factored using classical computing is RSA-768, a 232 digit semiprime. 
\subsection{Objectives}
Shor's algorithm can be adapted to solve elliptic curve discrete logarithm problems (ECDLP)~\cite{proos2003shor}, which is the set of non-inversible functions that are used in elliptic curve cryptography. These cryptographic schemes rely on the fact that given a point $Q$ on an elliptic curve, which is some multiple of an original point, $P$, such that $Q = [m]P$, it is very hard to calculate $m$ on classical computers. 
It has been estimated that it would require a quantum computer with approximately 1000 qubits to break a 160 bit elliptic curve cryptographic key, and approximately 2000 qubits to break the equivalently secure 1024 bit RSA~\cite{proos2003shor}.
This project builds on previous work and provides an estimate for the number of quantum gates required to break the discrete logarithm problem for hyperelliptic curves of genus 2.