As described in~\cite{roetteler2017quantum}, to calculate exact gate costs would require simulation of the quantum circuits involved, since different underlying bit patterns of the prime $p$ which the field is defined over give rise to different quantum circuits, so the actual estimation will be a interpolation based on points gathered for specific values of $p$. This does not affect the leading coefficient of the estimation, however, so we are able to provide leading coefficient estimates for each of the algorithms described in Table \ref{table:arithmeticCosts}, and the actual estimate will be bounded by this leading order coefficient. The results of this are displayed in Table \ref{table:totalGateCosts}. Unsuprisingly, the cost for kummer surface arithmetic is the lowest, and the cost for the affine coordinate addition was the highest, but this is purely focusing on the complexity of the quantum aspect of the algorithm. The pre and post processing are not taken into account, which can affect the time complexity for the whole algorithm drastically.
\begin{table}[!htb]
\resizebox{\textwidth}{!}{
\begin{tabular}{|c|c|c|c|c|}
\hline
Operation        & \begin{tabular}[c]{@{}c@{}}Toffoli Gates for \\ Addition\end{tabular} & \begin{tabular}[c]{@{}c@{}}Gates for Shor's \\ Algorithm\end{tabular} & \begin{tabular}[c]{@{}c@{}}Qubits for \\ Addition\end{tabular} & \begin{tabular}[c]{@{}c@{}}Qubits for Shor's \\ Algorithm\end{tabular} \\ \hline
$N+N=P$          & $1632n^2\log_2n$                                                      & $3264n^3\log_2n$                                                      & $5n+4$                                                         & $9n+5$                                                                 \\
$N+P=P$          & $1632n^2\log_2n$                                                      & $3264n^3\log_2n$                                                      & $5n+4$                                                         & $9n+5$                                                                 \\
$N+N=N$          & $1504n^2\log_2n$                                                      & $3008n^3\log_2n$                                                      & $5n+4$                                                         & $9n+5$                                                                 \\
$N+P=N$          & $1536n^2\log_2n$                                                      & $3072n^3\log_2n$                                                      & $5n+4$                                                         & $9n+5$                                                                 \\
$P+P=P$          & $1504n^2\log_2n$                                                      & $3008n^3\log_2n$                                                      & $5n+4$                                                         & $9n+5$                                                                 \\
$P+P=N$          & $1408n^2\log_2n$                                                      & $2816n^3\log_2n$                                                      & $5n+4$                                                         & $9n+5$                                                                 \\
$A+N=P$          & $1280n^2\log_2n$                                                      & $2560n^3\log_2n$                                                      & $5n+4$                                                         & $9n+5$                                                                 \\
$A+P=P$          & $1280n^2\log_2n$                                                      & $2560n^3\log_2n$                                                      & $5n+4$                                                         & $9n+5$                                                                 \\
$A+N=N$          & $1152n^2\log_2n$                                                      & $2304n^3\log_2n$                                                      & $5n+4$                                                         & $9n+5$                                                                 \\
$A+P=N$          & $1184n^2\log_2n$                                                      & $2368n^3\log_2n$                                                      & $5n+4$                                                         & $9n+5$                                                                 \\
$A+A=A$          & $736n^2\log_2n$                                                       & $1472n^3\log_2n$                                                      & $7n+2\lceil\log_2n\rceil+9$                                         & $11n+2\lceil\log_2n\rceil+10$                                                \\
$k(A)+k(A)=k(A)[1]$ & $992n^2\log_2n$                                                       & $1984n^3\log_2n$                                                       & $5n+4$                                                         & $9n+5$                                                                
\\
$k(A)+k(A)=k(A)[2]$ & $480n^2\log_2n$                                                       & $960n^3\log_2n$                                                       & $5n+4$                                                         & $9n+5$                                                                
\\ \hline
\end{tabular}}
\caption{Gate and qubit costs for algorithms with regard to $n$, the bit-size of the prime $p$ for which a hyperelliptic curve is defined over}
\label{table:totalGateCosts}
\end{table}
\subsection{Comparison of results}
As shown in Table \ref{table:totalGateCosts}, the gate costs when using projective and new coordinates are larger than affine coordinates, this being due to the large amount of extra multiplication and subtraction operations required in the algorithms. Usually in classical computing, these models would provide a much more efficient implementation, however due to the complex natures of the circuits required to implement the reversible operations in quantum computing, they provide a more inefficient implmentation. They do, however, require fewer total qubits and so a tradeoff is presented between hardware scale and circuit size. The two results for both variations of the Kummer surface representation are shown at the bottom of the table, with the representation containing the conversion from affine to kummer having a larger overall estimate than the normal affine coordinate representation, but with fewer required qubits. The second variation where the conversion is pre computed has a lower overall gate cost and a lower number of required qubits, however, as discussed in the previous section, this may not be feasible to implement. The number of toffoli gates for one divisor addition was calculated by multiplying the number of each type of operation by the gate cost for the equivalent circuit described in~\cite{roetteler2017quantum}. This estimate could then be used to calculate the first coefficient for the polynomial that bounds the overall gates required for Shor's algorithm by multiplying by $2n$, since the controlled divisor addition is iterated $2n$. The number of qubits required can be calculated as the most costly operation in terms of qubits, added to 4n+1, which is the number of ancillary qubits required for the operations. It is 4n in the case of hyperelliptic curves as opposed to the 2n of elliptic curves due to the way the divisor is stored: $[u_{1},u_{0},v_{1},v_{0}]$.
