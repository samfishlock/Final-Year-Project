To make a comparison between our results and the results found in~\cite{roetteler2017quantum}, we must take into account the fact that an equivalently secure hyperelliptic curve cryptosystem requires approximately half of an elliptic curve cryptosystem's prime in terms of its bit-size~\cite{cryptoeprint:2012:670}. Given this, we can divide our results by 2 to more accurately compare the complexities of the two different circuits. In this case, we have a result of  $448n^3\log_2n$ for elliptic curves, and $480n^3\log_2n$ in the case of hyperelliptic curves when using the Kummer surface model, or $736n^3\log_2n$ in case of affine coordinates, thus even with the less complex bit-size prime, the estimates are still higher than that of elliptic curves. This is ultimately due to the more complex nature of hyperelliptic curve arithemtic, with even the most stripped down version requiring 15 multiplications and 2 subtractions, as opposed to the 4 multipliers, 2 squarers and 4 inversions of elliptic curves. Due to the results gained in~\cite{roetteler2017quantum}, we can also conclude that solving the discrete logarithm problem for hyperelliptic curves is easier to compute on a quantum computer than solving the factorisation problem for RSA.
\subsection{Possible Extensions}
In order to calculate a more accurate estimate for the number of gates required for a divisor addition, we would have to simulate the circuit in a quantum circuit simulator such as {LIQU}i|>~\cite{1402.4467} for different bit-sizes of $p$, and use interpolation to calculate a curve of best fit of these points, as per the method described in~\cite{roetteler2017quantum}.
Another possible extension could be to verify whether we can calculate the conversion from affine coordinates to Kummer surface coordinates in pre processing to avoid complexity in the quantum circuit. We can do this using the same quantum simulation software, as we can verify that our results match those gained using the conventional method using affine coordinates.