To make a comparison between our results and the results found in~\cite{roetteler2017quantum}, we must take into account the fact that an equivalently secure hyperelliptic curve cryptosystem requires approximately half of an elliptic curve cryptosystem's prime in terms of its bit-size~\cite{cryptoeprint:2012:670}. Given this, we can divide our input bit-size by 2 to more accurately compare the complexities of the two different circuits. In terms of $n$, the bit-size of the prime, we have a result of  $448n^3\log_2n$ for elliptic curves, and $960n^3\log_2n$ in the case of hyperelliptic curves when using the Kummer surface model, or $1472n^3\log_2n$ in case of affine coordinates. As shown, the leading integer coefficient is much larger in hyperelliptic curves, this is ultimately due to the more complex nature of hyperelliptic curve arithemtic, with even the most stripped down version requiring 15 multiplications and 2 subtractions, as opposed to the 4 multipliers, 2 squarers and 4 inversions of elliptic curves. Table \ref{table:circuitSizeComparison}, adapted from~\cite{roetteler2017quantum}, compares the circuit size and complexity for various bit-size primes of RSA, elliptic curves and hyperelliptic curves.
\begin{table}[!htb]
\resizebox{\linewidth}{!}{
\begin{tabular}{|c|c|c|c|c|c|c|c|c|c|c|}
\hline
\multicolumn{3}{|c|}{Factoring of RSA modulus N}                                          & \multicolumn{3}{c|}{ECDLP in $E(\mathbb{F}_p)$}                                               & \multicolumn{5}{c|}{HECDLP in $E(\mathbb{F}_p)$}                                                                                                                                                                                                                                                                              \\ \hline
\begin{tabular}[c]{@{}c@{}}$\log_2(N)$ \\ bits\end{tabular} & Qubits & Toffoli gates      & \begin{tabular}[c]{@{}c@{}}$\log_2(p)$ \\ bits\end{tabular} & Qubits & Toffoli gates          & \begin{tabular}[c]{@{}c@{}}$\log_2(p)$ \\ bits\end{tabular} & \begin{tabular}[c]{@{}c@{}}Qubits \\ (Affine)\end{tabular} & \begin{tabular}[c]{@{}c@{}}Qubits \\ (Kummer)\end{tabular} & \begin{tabular}[c]{@{}c@{}}Toffoli gates \\ (Affine)\end{tabular} & \begin{tabular}[c]{@{}c@{}}Toffoli gates \\ (Kummer)\end{tabular} \\ \hline
512                                                         & 1026   & $6.41\cdot10^{10}$ & 110                                                         & 1014   & $\sim4.04\cdot10^{9}$  & 55                                                          & 627                                                        & 500                                                        & $\sim1.42\cdot10^{9}$                                             & $\sim9.23\cdot10^{8}$                                             \\
1024                                                        & 2050   & $5.81\cdot10^{11}$ & 160                                                         & 1466   & $\sim1.34\cdot10^{10}$ & 80                                                          & 903                                                        & 725                                                        & $\sim4.76\cdot10^{9}$                                             & $\sim3.11\cdot10^{9}$                                             \\
-                                                           & -      & -                  & 192                                                         & 1754   & $\sim2.41\cdot10^{10}$ & 96                                                          & 1080                                                       & 869                                                        & $\sim8.58\cdot10^{9}$                                             & $\sim5.59\cdot10^{9}$                                             \\
2048                                                        & 4098   & $5.20\cdot10^{12}$ & 224                                                         & 2042   & $\sim3.93\cdot10^{10}$ & 112                                                         & 1256                                                       & 1013                                                       & $\sim1.41\cdot10^{10}$                                            & $\sim9.18\cdot10^{9}$                                             \\
3072                                                        & 6146   & $1.86\cdot10^{13}$ & 256                                                         & 2330   & $\sim6.01\cdot10^{10}$ & 128                                                         & 1432                                                       & 1157                                                       & $\sim2.16\cdot10^{10}$                                            & $\sim1.41\cdot10^{10}$                                            \\
7680                                                        & 15362  & $3.30\cdot10^{14}$ & 384                                                         & 3484   & $\sim2.18\cdot10^{11}$ & 192                                                         & 2138                                                       & 1733                                                       & $\sim7.90\cdot10^{10}$                                            & $\sim5.15\cdot10^{10}$                                            \\
15360                                                       & 30722  & $2.87\cdot10^{15}$ & 521                                                         & 4719   & $\sim5.72\cdot10^{11}$ & 260                                                         & 2887                                                       & 2345                                                       & $\sim2.08\cdot10^{11}$                                            & $\sim1.35\cdot10^{11}$                                            \\ \hline
\end{tabular}}
\caption{Comparison of estimations of circuit sizes between different cryptographic schemes}
\label{table:circuitSizeComparison}
\end{table}
As shown in the table, although the leading integer coefficient is much larger for hyperelliptic curves, the overall circuit complexity is lower, due to the smaller bit-size inputs. The number of qubits required in the circuits for hyperelliptic curves is approximately half of that for elliptic curves.
\subsection{Possible Extensions}
In order to calculate a more accurate estimate for the number of gates required for a divisor addition, we would have to simulate the circuit in a quantum circuit simulator such as {LIQU}i|>~\cite{1402.4467} for different bit-sizes of $p$, and use interpolation to calculate a curve of best fit of these points, as per the method described in~\cite{roetteler2017quantum}.
Another possible extension could be to verify whether we can calculate the conversion from affine coordinates to Kummer surface coordinates in pre processing to avoid complexity in the quantum circuit. We can do this using the same quantum simulation software, as we can verify that our results match those gained using the conventional method using affine coordinates.